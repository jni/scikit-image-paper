%---------
% Preamble
%---------
\documentclass[fleqn,12pt]{wlpeerj}
\usepackage{fixltx2e} % LaTeX patches, \textsubscript
\usepackage{cmap} % fix search and cut-and-paste in Acrobat
\usepackage{ifthen}
\usepackage[T1]{fontenc}
\usepackage[utf8]{inputenc}
\usepackage{hyperref}
\definecolor{orange}{cmyk}{0,0.4,0.8,0.2}
\definecolor{darkorange}{rgb}{.71,0.21,0.01}
\definecolor{darkblue}{rgb}{.01,0.21,0.71}
\definecolor{darkgreen}{rgb}{.1,.52,.09}
\hypersetup{pdftex,  % needed for pdflatex
  breaklinks=true,  % so long urls are correctly broken across lines
  colorlinks=true,
  urlcolor=blue,
  linkcolor=darkblue,
  citecolor=darkgreen,
  }

\setcounter{secnumdepth}{0}
\graphicspath{{../output/skimage/}}

% hyperlinks:
\ifthenelse{\isundefined{\hypersetup}}{
  \usepackage[colorlinks=true,linkcolor=blue,urlcolor=blue]{hyperref}
  \urlstyle{same} % normal text font (alternatives: tt, rm, sf)
}{}

\usepackage{caption}
\usepackage{subcaption}
\usepackage{float}

% Add line numbers globally for reviewers
\usepackage{lineno}
\linenumbers

\usepackage[gobble=auto]{pythontex}  % Must better syntax highlighting than listings

\PassOptionsToPackage{pdftex}{graphicx}
\usepackage{graphicx}  % Simplify graphics handling for figures


%-------------------------
% Begin scikit-image paper
%-------------------------

\begin{document}

\title{scikit-image: Image processing in Python}

% Author list affiliations
\author[1,2]{Stéfan van der Walt}
\affil[1]{Corresponding author: \protect\href{mailto:stefan@sun.ac.za}{stefan@sun.ac.za}}
\affil[2]{Stellenbosch University,
          Stellenbosch, South Africa}
\author[3]{Johannes L. Schönberger}
\affil[3]{Department of Computer Science,
          University of North Carolina at Chapel Hill,
          Chapel Hill, NC 27599, USA}
\author[4]{Juan Nunez-Iglesias}
\affil[4]{Victorian Life Sciences Computation Initiative,
          Carlton, VIC, 3010, Australia}
\author[5]{François Boulogne}
\affil[5]{Department of Mechanical and Aerospace Engineering,
          Princeton University,
          Princeton, New Jersey 08544, USA}
\author[6]{Joshua D. Warner}
\affil[6]{Department of Biomedical Engineering,
          Mayo Clinic,
          Rochester, Minnesota 55905, USA}
\author[7]{Neil Yager}
\affil[7]{AICBT Ltd,
          Oxford, UK}
\author[8]{Emmanuelle Gouillart}
\affil[8]{Joint Unit CNRS / Saint-Gobain,
          Cavaillon, France}
\author[9]{Tony Yu}
\affil[9]{Enthought Inc.,
          Austin, TX, USA}
\author[10]{the scikit-image contributors}
\affil[10]{\url{https://github.com/scikit-image/scikit-image/graphs/contributors}}

% Key words to associate with the article
\keywords{image processing, reproducible research, education, visualization}

%---------
% Abstract
%---------

\begin{abstract}
  scikit-image is an image processing library that implements algorithms and utilities for use in research, education and industry applications. It is released under the liberal Modified BSD open source license, provides a well-documented API in the Python programming language, and is developed by an active, international team of collaborators. In this paper we highlight the advantages of open source to achieve the goals of the scikit-image library, and we showcase several real-world image processing applications that use scikit-image. More information can be found on the project homepage, \url{http://scikit-image.org}.
\end{abstract}

\flushbottom
\maketitle
\thispagestyle{empty}

%-----------------------------------------------------------------
% Sections of the paper are included in separate files with \input
%-----------------------------------------------------------------

% Introduction
\input{introduction}

% Getting started
%----------------
% Getting Started
%----------------

\section*{Getting started}
  \label{sec:getting-started}

  One of the main goals of scikit-image is to make it easy for any user to get started quickly--especially users already familiar with Python's scientific tools. To that end, the basic image is just a standard NumPy array, which exposes pixel data directly to the user. A new user can simply the load an image from disk (or use one of scikit-image's sample images), process that image with one or more image filters, and quickly display the results:

  % Bring in correctly syntax highlighted example
  \begin{pyverbatim}
    from skimage import data, io, filter

    image = data.coins()  # or any NumPy array!
    edges = filter.sobel(image)
    io.imshow(edges)
  \end{pyverbatim}

  \begin{figure}
    \includegraphics[width=\columnwidth]{getting_started.eps}

    \caption[Getting started figure]{\label{fig:gettingstarted}Illustration of several functions available in scikit-image: adaptive threshold, local maxima, edge detection and labels. The use of NumPy arrays as our data container also enables the use of NumPy's built-in \texttt{histogram} function.}
  \end{figure}

  The above demonstration loads \texttt{data.coins}, an example image shipped with scikit-image.  For a more complete example, we import NumPy for array manipulation and matplotlib for plotting \citep{numpy,matplotlib}.  At each step, we add the picture or the plot to a matplotlib figure shown in Figure~\ref{fig:gettingstarted}.

  \begin{pyverbatim}
    import numpy as np
    import matplotlib.pyplot as plt

    # Load a small section of the image.
    image = data.coins()[0:95, 70:370]

    fig, axes = plt.subplots(ncols=2, nrows=3,
                             figsize=(8, 4))
    ax0, ax1, ax2, ax3, ax4, ax5  = axes.flat
    ax0.imshow(image, cmap=plt.cm.gray)
    ax0.set_title('Original', fontsize=24)
    ax0.axis('off')
  \end{pyverbatim}

  Since the image is represented by a NumPy array, we can easily perform operations such as building an histogram of the intensity values.

  \begin{pyverbatim}
    # Histogram.
    values, bins = np.histogram(image,
                                bins=np.arange(256))

    ax1.plot(bins[:-1], values, lw=2, c='k')
    ax1.set_xlim(xmax=256)
    ax1.set_yticks([0, 400])
    ax1.set_aspect(.2)
    ax1.set_title('Histogram', fontsize=24)
  \end{pyverbatim}

  To divide the foreground and background, we threshold the image to produce a binary image.  Several threshold algorithms are available. Here, we employ \linebreak\texttt{filter.threshold\_adaptive} where the threshold value is the weighted mean for the local neighborhood of a pixel.

  \begin{pyverbatim}
    # Apply threshold.
    from skimage.filter import threshold_adaptive

    bw = threshold_adaptive(image, 95, offset=-15)

    ax2.imshow(bw, cmap=plt.cm.gray)
    ax2.set_title('Adaptive threshold', fontsize=24)
    ax2.axis('off')
  \end{pyverbatim}

  We can easily detect interesting features, such as local maxima and edges. The function \texttt{feature.peak\_local\_max} can be used to return the coordinates of local maxima in an image.

  \begin{pyverbatim}
    # Find maxima.
    from skimage.feature import peak_local_max

    coordinates = peak_local_max(image, min_distance=20)

    ax3.imshow(image, cmap=plt.cm.gray)
    ax3.autoscale(False)
    ax3.plot(coordinates[:, 1],
             coordinates[:, 0], c='r.')
    ax3.set_title('Peak local maxima', fontsize=24)
    ax3.axis('off')
  \end{pyverbatim}

  Next, a Canny filter (\texttt{filter.canny}) \citep{Canny} detects the edge of each coin.

  \begin{pyverbatim}
    # Detect edges.
    from skimage import filter

    edges = filter.canny(image, sigma=3,
                         low_threshold=10,
                         high_threshold=80)

    ax4.imshow(edges, cmap=plt.cm.gray)
    ax4.set_title('Edges', fontsize=24)
    ax4.axis('off')
  \end{pyverbatim}

  Then, we attribute to each coin a label (\texttt{morphology.label}) that can be used to extract a sub-picture. Finally, physical information such as the position, area, eccentricity, perimeter, and moments can be extracted using \texttt{measure.regionprops}.

  \begin{pyverbatim}
    # Label image regions.
    from skimage.measure import regionprops
    import matplotlib.patches as mpatches
    from skimage.morphology import label

    label_image = label(edges)

    ax5.imshow(image, cmap=plt.cm.gray)
    ax5.set_title('Labeled items', fontsize=24)
    ax5.axis('off')

    for region in regionprops(label_image):
        # Draw rectangle around segmented coins.
        minr, minc, maxr, maxc = region.bbox
        rect = mpatches.Rectangle((minc, minr),
                                  maxc - minc,
                                  maxr - minr,
                                  fill=False,
                                  edgecolor='red',
                                  linewidth=2)
        ax5.add_patch(rect)

    plt.tight_layout()
    plt.show()
  \end{pyverbatim}

  scikit-image thus makes it possible to perform sophisticated image processing tasks with only a few function calls.


% Library contents
\input{library_contents}

% Data format and pipelining
\input{data_format_pipelining}

% Development practices
\input{development_practices}

% Large section broken into separate files for each sub-section
\section*{Usage examples}
  \label{sec:usage-examples}

  % Usage in research
  %-------------------------
% Usage Examples: Research
%-------------------------

  \subsection*{Research}
    \label{sub:research}

    Often, a disproportionately large component of research involves dealing with various image data-types, color representations, and file format conversion. scikit-image offers robust tools for converting between image data-types \citep{DirectX,OpenGL,GraphicsGemsI} and to do file input/output (I/O) operations.  Our purpose is to allow investigators to focus their time on research, instead of expending effort on mundane low-level tasks.

    The package includes a number of algorithms with broad applications across image processing research, from computer vision to medical image analysis. We refer the reader to the current API documentation for a full listing of current capabilities\footnote{\url{http://scikit-image.org/docs/dev}, Accessed 2014-03-30}. In this section we illustrate two real-world usage examples of scikit-image in scientific research.

    First, we consider the analysis of a large stack of images, each representing drying droplets containing nanoparticles (see Figure~\ref{fig:cracks}). As the drying proceeds, cracks propagate from the edge of the drop to its center. The aim is to understand crack patterns by collecting statistical information about their positions, as well as their time and order of appearance. To improve the speed at which data is processed, each experiment, constituting an image stack, is automatically analysed without human intervention. The contact line is detected by a circular Hough transform (\texttt{transform.hough\_circle}) providing the drop radius and its center. Then, a smaller concentric circle is drawn (\texttt{draw.circle\_perimeter}) and used as a mask to extract intensity values from the image. Repeating the process on each image in the stack, collected pixels can be assembled to make a space-time diagram. As a result, a complex stack of images is reduced to a single image summarizing the underlying dynamic process.

    \begin{figure}[bht]
      \includegraphics[width=\columnwidth]{fig_cracks.png}

      \caption{scikit-image is used to track the propagation of cracks (black lines) in a drying colloidal droplet. The sequence of pictures shows the temporal evolution of the system with the drop contact line, in green, detected by the Hough transform and the circle, in white, used to extract an annulus of pixel intensities.  The result shown illustrates the angular position of cracks and their time of appearance. \label{fig:cracks}}
    \end{figure}

    Next, in regenerative medicine research, scikit-image is used to monitor the regeneration of spinal cord cells in zebrafish embryos (Figure \ref{fig:profile}). This process has important implications for the treatment of spinal cord injuries in humans \citep{Bhatt04,Thuret06}.

    To understand how spinal cords regenerate in these animals, injured cords are subjected to different treatments. Neuronal precursor cells (labeled green in Figure \ref{fig:profile}, left panel) are normally uniformly distributed across the spinal cord. At the wound site, they have been removed. We wish to monitor the arrival of new cells at the wound site over time. In Figure \ref{fig:profile}, we see an embryo two days after wounding, with precursor cells beginning to move back into the wound site (the site of minimum fluorescence). The \texttt{measure.profile\_line} function measures the fluorescence along the cord, directly proportional to the number of cells. We can thus monitor the recovery process and determine which treatments prevent or accelerate recovery.

    \begin{figure*}[bht]

      \includegraphics[width=\columnwidth]{fig-lesion.eps}

      \caption{The \texttt{measure.profile\_line} function being used to track recovery in spinal cord injuries. (a): an image of fluorescently-labeled nerve cells in an injured zebrafish embryo. (b): the automatically determined region of interest. The SciPy library was used to determine the region extent \citep{scipy,scipylib}, and functions from the scikit-image \texttt{draw} module were used to draw it. (c): the image intensity along the line of interest, averaged over the displayed width. \label{fig:profile}}
    \end{figure*}


  % Usage in education
  \input{usage_education}

  % Usage in industry
  \input{usage_industry}

% Example: image registration and stitching
\input{panorama_example}

% Discussion
\input{discussion}

% Conclusion
\input{conclusion}

% Acknowledgements
\input{acknowledgements}

% Finally, create the bibliograpy
\bibliography{skimage}

\end{document}
